% Options for packages loaded elsewhere
\PassOptionsToPackage{unicode}{hyperref}
\PassOptionsToPackage{hyphens}{url}
%
\documentclass[
  ignorenonframetext,
]{beamer}
\usepackage{pgfpages}
\setbeamertemplate{caption}[numbered]
\setbeamertemplate{caption label separator}{: }
\setbeamercolor{caption name}{fg=normal text.fg}
\beamertemplatenavigationsymbolsempty
% Prevent slide breaks in the middle of a paragraph
\widowpenalties 1 10000
\raggedbottom
\setbeamertemplate{part page}{
  \centering
  \begin{beamercolorbox}[sep=16pt,center]{part title}
    \usebeamerfont{part title}\insertpart\par
  \end{beamercolorbox}
}
\setbeamertemplate{section page}{
  \centering
  \begin{beamercolorbox}[sep=12pt,center]{part title}
    \usebeamerfont{section title}\insertsection\par
  \end{beamercolorbox}
}
\setbeamertemplate{subsection page}{
  \centering
  \begin{beamercolorbox}[sep=8pt,center]{part title}
    \usebeamerfont{subsection title}\insertsubsection\par
  \end{beamercolorbox}
}
\AtBeginPart{
  \frame{\partpage}
}
\AtBeginSection{
  \ifbibliography
  \else
    \frame{\sectionpage}
  \fi
}
\AtBeginSubsection{
  \frame{\subsectionpage}
}
\usepackage{amsmath,amssymb}
\usepackage{iftex}
\ifPDFTeX
  \usepackage[T1]{fontenc}
  \usepackage[utf8]{inputenc}
  \usepackage{textcomp} % provide euro and other symbols
\else % if luatex or xetex
  \usepackage{unicode-math} % this also loads fontspec
  \defaultfontfeatures{Scale=MatchLowercase}
  \defaultfontfeatures[\rmfamily]{Ligatures=TeX,Scale=1}
\fi
\usepackage{lmodern}
\ifPDFTeX\else
  % xetex/luatex font selection
\fi
% Use upquote if available, for straight quotes in verbatim environments
\IfFileExists{upquote.sty}{\usepackage{upquote}}{}
\IfFileExists{microtype.sty}{% use microtype if available
  \usepackage[]{microtype}
  \UseMicrotypeSet[protrusion]{basicmath} % disable protrusion for tt fonts
}{}
\makeatletter
\@ifundefined{KOMAClassName}{% if non-KOMA class
  \IfFileExists{parskip.sty}{%
    \usepackage{parskip}
  }{% else
    \setlength{\parindent}{0pt}
    \setlength{\parskip}{6pt plus 2pt minus 1pt}}
}{% if KOMA class
  \KOMAoptions{parskip=half}}
\makeatother
\usepackage{xcolor}
\newif\ifbibliography
\usepackage{color}
\usepackage{fancyvrb}
\newcommand{\VerbBar}{|}
\newcommand{\VERB}{\Verb[commandchars=\\\{\}]}
\DefineVerbatimEnvironment{Highlighting}{Verbatim}{commandchars=\\\{\}}
% Add ',fontsize=\small' for more characters per line
\usepackage{framed}
\definecolor{shadecolor}{RGB}{248,248,248}
\newenvironment{Shaded}{\begin{snugshade}}{\end{snugshade}}
\newcommand{\AlertTok}[1]{\textcolor[rgb]{0.94,0.16,0.16}{#1}}
\newcommand{\AnnotationTok}[1]{\textcolor[rgb]{0.56,0.35,0.01}{\textbf{\textit{#1}}}}
\newcommand{\AttributeTok}[1]{\textcolor[rgb]{0.13,0.29,0.53}{#1}}
\newcommand{\BaseNTok}[1]{\textcolor[rgb]{0.00,0.00,0.81}{#1}}
\newcommand{\BuiltInTok}[1]{#1}
\newcommand{\CharTok}[1]{\textcolor[rgb]{0.31,0.60,0.02}{#1}}
\newcommand{\CommentTok}[1]{\textcolor[rgb]{0.56,0.35,0.01}{\textit{#1}}}
\newcommand{\CommentVarTok}[1]{\textcolor[rgb]{0.56,0.35,0.01}{\textbf{\textit{#1}}}}
\newcommand{\ConstantTok}[1]{\textcolor[rgb]{0.56,0.35,0.01}{#1}}
\newcommand{\ControlFlowTok}[1]{\textcolor[rgb]{0.13,0.29,0.53}{\textbf{#1}}}
\newcommand{\DataTypeTok}[1]{\textcolor[rgb]{0.13,0.29,0.53}{#1}}
\newcommand{\DecValTok}[1]{\textcolor[rgb]{0.00,0.00,0.81}{#1}}
\newcommand{\DocumentationTok}[1]{\textcolor[rgb]{0.56,0.35,0.01}{\textbf{\textit{#1}}}}
\newcommand{\ErrorTok}[1]{\textcolor[rgb]{0.64,0.00,0.00}{\textbf{#1}}}
\newcommand{\ExtensionTok}[1]{#1}
\newcommand{\FloatTok}[1]{\textcolor[rgb]{0.00,0.00,0.81}{#1}}
\newcommand{\FunctionTok}[1]{\textcolor[rgb]{0.13,0.29,0.53}{\textbf{#1}}}
\newcommand{\ImportTok}[1]{#1}
\newcommand{\InformationTok}[1]{\textcolor[rgb]{0.56,0.35,0.01}{\textbf{\textit{#1}}}}
\newcommand{\KeywordTok}[1]{\textcolor[rgb]{0.13,0.29,0.53}{\textbf{#1}}}
\newcommand{\NormalTok}[1]{#1}
\newcommand{\OperatorTok}[1]{\textcolor[rgb]{0.81,0.36,0.00}{\textbf{#1}}}
\newcommand{\OtherTok}[1]{\textcolor[rgb]{0.56,0.35,0.01}{#1}}
\newcommand{\PreprocessorTok}[1]{\textcolor[rgb]{0.56,0.35,0.01}{\textit{#1}}}
\newcommand{\RegionMarkerTok}[1]{#1}
\newcommand{\SpecialCharTok}[1]{\textcolor[rgb]{0.81,0.36,0.00}{\textbf{#1}}}
\newcommand{\SpecialStringTok}[1]{\textcolor[rgb]{0.31,0.60,0.02}{#1}}
\newcommand{\StringTok}[1]{\textcolor[rgb]{0.31,0.60,0.02}{#1}}
\newcommand{\VariableTok}[1]{\textcolor[rgb]{0.00,0.00,0.00}{#1}}
\newcommand{\VerbatimStringTok}[1]{\textcolor[rgb]{0.31,0.60,0.02}{#1}}
\newcommand{\WarningTok}[1]{\textcolor[rgb]{0.56,0.35,0.01}{\textbf{\textit{#1}}}}
\usepackage{graphicx}
\makeatletter
\def\maxwidth{\ifdim\Gin@nat@width>\linewidth\linewidth\else\Gin@nat@width\fi}
\def\maxheight{\ifdim\Gin@nat@height>\textheight\textheight\else\Gin@nat@height\fi}
\makeatother
% Scale images if necessary, so that they will not overflow the page
% margins by default, and it is still possible to overwrite the defaults
% using explicit options in \includegraphics[width, height, ...]{}
\setkeys{Gin}{width=\maxwidth,height=\maxheight,keepaspectratio}
% Set default figure placement to htbp
\makeatletter
\def\fps@figure{htbp}
\makeatother
\setlength{\emergencystretch}{3em} % prevent overfull lines
\providecommand{\tightlist}{%
  \setlength{\itemsep}{0pt}\setlength{\parskip}{0pt}}
\setcounter{secnumdepth}{-\maxdimen} % remove section numbering
\ifLuaTeX
  \usepackage{selnolig}  % disable illegal ligatures
\fi
\IfFileExists{bookmark.sty}{\usepackage{bookmark}}{\usepackage{hyperref}}
\IfFileExists{xurl.sty}{\usepackage{xurl}}{} % add URL line breaks if available
\urlstyle{same}
\hypersetup{
  pdftitle={soc\_stat\_demo},
  pdfauthor={Eli Lin},
  hidelinks,
  pdfcreator={LaTeX via pandoc}}

\title{soc\_stat\_demo}
\author{Eli Lin}
\date{2023-10-24}

\begin{document}
\frame{\titlepage}

\begin{frame}{learning resources and mindset}
\protect\hypertarget{learning-resources-and-mindset}{}
\begin{itemize}
\tightlist
\item
  \href{https://stackoverflow.com/}{Stackoverflow}
\item
  \href{https://www.edx.org/learn/r-programming/harvard-university-data-science-r-basics?index=product\&queryID=e2a15d51e39a3dd02da1dc5ea276aaef\&position=1\&results_level=first-level-results\&term=r+basic\&objectID=course-91f52ef3-fa3f-4934-9d19-8d5a32635cd4\&campaign=Data+Science\%3A+R+Basics\&source=edX\&product_category=course\&placement_url=https\%3A\%2F\%2Fwww.edx.org\%2Fsearchhttps://}{Edx}
\item
  \href{https://github.com/ujjwalkarn/DataScienceR}{github上的學習資源整理}
\item
  \href{https://biocorecrg.github.io/CRG_RIntroduction/}{隨便找的筆記}
\item
  \href{https://bookdown.org/b08302310/R_learning_notes/}{隨便找的筆記2}
\item
  \href{https://app.datacamp.com/learn/courses/free-introduction-to-r}{互動式練習}
\item
  \href{https://app.datacamp.com/learn/courses/introduction-to-the-tidyverse}{tidyverse}
\item
  \href{https://strengejacke.github.io/sjPlot/}{sjplot}
\item
  \href{https://bookdown.org/tonykuoyj/eloquentr/}{台大郭耀仁教授}
  \#注意多層函數寫法不同
\item
  \href{https://statsandr.com/blog/descriptive-statistics-in-r/}{Descriptive
  statistics in R}
\item
  \href{https://support.posit.co/hc/en-us/articles/201057987-Quick-list-of-useful-R-packages}{Quick
  list of useful R packages}
\item
  \href{https://support.rstudio.com/hc/en-us/articles/200711853-Keyboard-Shortcuts}{Rstudio
  hot keys}
\item
  \textbf{不要去什麼hahow之類的給人當韭菜割}
\item
  學習程式語言的心態
\end{itemize}
\end{frame}

\begin{frame}{installation, enviroment, and setting}
\protect\hypertarget{installation-enviroment-and-setting}{}
\begin{itemize}
\tightlist
\item
  R是程式語言, Rstudio是IDE
\item
  \href{https://cran.csie.ntu.edu.tw/}{CRAN NTU mirror}
\item
  \href{https://posit.co/downloads/}{Rstudio}
\item
  environment: grid
\item
  不建議保留.Rdata
\item
  new project and setting directory
\end{itemize}
\end{frame}

\begin{frame}{hot keys}
\protect\hypertarget{hot-keys}{}
\begin{itemize}
\tightlist
\item
  ctrl + L clear console
\item
  ctrl + alt + R / crtl + shift + enter
\item
  tab 操作snippets
\item
  ctrl + tab
\item
  alt + - 輸入賦值符號''\textless-''
\item
  ctrl + shift + F11 重啟R, detach套件, 並清除環境
\end{itemize}
\end{frame}

\begin{frame}{style, code etiquette}
\protect\hypertarget{style-code-etiquette}{}
\begin{itemize}
\tightlist
\item
  請讓人好讀
\item
  \href{https://bookdown.org/tonykuoyj/eloquentr/styleguide.html}{style}
\item
  \href{https://style.tidyverse.org/syntax.html}{style, code etiquette}
\end{itemize}
\end{frame}

\begin{frame}{packages}
\protect\hypertarget{packages}{}
\begin{itemize}
\tightlist
\item
  tidyverse 整合套件,可以用這個去查教學

  \begin{itemize}
  \tightlist
  \item
    ggplot2 作圖
  \item
    dplyr 資料整理
  \item
    tidyr 表格轉換
  \item
    readr 資料載入
  \item
    tibble 另一種資料格式
  \end{itemize}
\item
  sjplot

  \begin{itemize}
  \tightlist
  \item
    sjlabelled:讀取處理有標籤格式資料檔與變數
  \item
    sjPlot:快速製表與製圖
  \item
    sjmisc:效率高的變數描述與編碼
  \item
    Sjstats:快速提供模型統計量的計算
  \end{itemize}
\item
  haven 讀取spss, stat檔案
\item
  readrxl 讀取excle檔案
\end{itemize}
\end{frame}

\begin{frame}{zz}
\protect\hypertarget{zz}{}
\begin{itemize}
\tightlist
\item
  Bullet 1
\item
  Bullet 2
\item
  Bullet 3
\end{itemize}
\end{frame}

\begin{frame}[fragile]{Slide with R Output}
\protect\hypertarget{slide-with-r-output}{}
\begin{Shaded}
\begin{Highlighting}[]
\FunctionTok{summary}\NormalTok{(cars)}
\end{Highlighting}
\end{Shaded}

\begin{verbatim}
##      speed           dist       
##  Min.   : 4.0   Min.   :  2.00  
##  1st Qu.:12.0   1st Qu.: 26.00  
##  Median :15.0   Median : 36.00  
##  Mean   :15.4   Mean   : 42.98  
##  3rd Qu.:19.0   3rd Qu.: 56.00  
##  Max.   :25.0   Max.   :120.00
\end{verbatim}
\end{frame}

\begin{frame}{Slide with Plot}
\protect\hypertarget{slide-with-plot}{}
\includegraphics{rmdtest_files/figure-beamer/pressure-1.pdf}
\end{frame}

\end{document}
